%
%
%
\documentclass[10pt,a4paper,french]{article}
\usepackage{times,helvet}
\usepackage{ulem,vmargin}
\usepackage{graphicx}          
\usepackage[T1]{fontenc}
\usepackage{SIunits}
\usepackage[utf8]{inputenc}
\usepackage[french]{babel}
\usepackage{hyperref}
\usepackage{moreverb}
\setpapersize{A4}

\oddsidemargin 20mm
\evensidemargin 20mm
\marginparsep 0mm
\textwidth 170mm

\newcommand{\bc}{\begin{center}}
\newcommand{\ec}{\end{center}}
\newcommand{\bi}{\begin{itemize}}
\newcommand{\ei}{\end{itemize}}
\newcommand{\be}{\begin{enumerate}}
\newcommand{\ee}{\end{enumerate}}

\title{Extensions FITS et règles de gestion des acquisitions et traitements des séquences d'images en astrophotographie}
\author{https://github.com/euldulle/seqfits\\fm@dulle.fr}


\pagestyle{myheadings}
\markboth{}{ }

\begin{document}
\maketitle
\parindent=0cm
\parskip=0.3cm
\section*{Résumé et motivations}

(Ce document est principalement un document de travail orienté développement ;
pour l'observateur, retenir que le projet vise à rendre possible le pilotage des 
acquisitions et l'automatisation des traitements à partir d'un mimimum d'informations 
-voir les 2 lignes "observateur" du Makefile).

Pour un certain nombre de raisons, l'objet de base manipulé en astrophotographie amateur 
n'est plus une image mais une séquence d'images. C'est valable autant pour les images de
calibration que pour les images "signal" et cela impacte les processus d'acquisition 
comme ceux de traitement.

L'approche naturelle qui prévaut est celle consistant à coder dans le nom des fichiers
manipulés les informations nécessaires au traitement (type d'image, numéro dans la séquence...)
Cette approche souffre d'une part d'un manque de cadre systématique (2 logiciels d'acquisition
différents, ou 2 observateurs différents vont créer des fichiers aux noms différents pour
le même objet), et de la fragilité intrinsèque d'une telle approche (quand le simple renommage 
d'un fichier suffit à le rendre inexploitable, il est légitime de penser qu'il y a un 
problème conceptuel).

Dans cette approche la construction d'une image avec 4 ou 5 bandes de fréquence, et potentiellement
1, 2 ou 3 temps de pose différents est un casse-tête organisationnel, si ce n'est au moment 
de l'acquisition, en tous cas au moment du traitement : l'observateur se retrouve à devoir 
faire à la main des opérations (renommage, déplacement de fichiers, création de répertoire)
fasitdieuses et sans aucun intérêt. C'est d'autant plus paradoxal qu'on a sous la main 
des puissances de calcul... disons astronomiques.

Ce document propose une approche systématique de la construction d'une image finale (le projet) 
où toutes les informations pertinentes à la réalisation du projet sont encodées dans les 
fichiers FITS au moment de l'acquisition (ou de la conversion en fits pour les acquisitions APN).

L'objectif est de déterminer le plus petit ensemble d'informations que doit fournir
l'observateur pour que, moyennant quelques règles communes, logiciels d'acquisition et de traitement
puissent faire leur travail sans que l'observateur ait à intervenir sur les détails 
avant la production des images finales.

Le premier élément de la proposition est un Makefile dont l'objectif est la construction
d'une liste de tâches (de 2 types, acquisition ou traitement) à accomplir, et d'accompagner 
chacune de ces tâches des paramètres utiles à la traçabilité des images produites à chaque
étape.
Le Makefile proposé est générique, l'input observateur se limite au nom du projet et
aux composantes de l'image.

La traçabilité des images s'appuie sur des mots-clés (KEYWORDS) du standard FITS ; 3 sont 
déjà largement utilisés (IMAGETYP, FREQ, EXPTIME), et le quatrième, SEQNUM, doit permettre
de gérer l'aspect séquence. On peut aussi introduire un mot-clé PROJECT, mais il n'est pas 
strictement nécessaire.

Le deuxième élément c'est la liste des tâches (ACQ et PROC) à effectuer pour mener 
le projet à bien. Dans ce document les tâches sont décrites de manière symbolique, 
l'essentiel étant la présence des paramètres permettant d'assurer la traçabilité.

Le troisième élément c'est les règles à suivre par l'acquisition et le traitement pour
que le premier fasse ce qu'il faut pour que le deuxième puisse retrouver ses petits
sans rien demander à personne.

\section{Exemple symbolique, Makefile, génération des tâches}

Donc voilà le Makefile, suivi de la liste des tâches générées quand on
mouline ce Makefile par un simple make.

\verbatimtabinput{Makefile}
\bi
\item Le fichier projectname.tasks (toutes les tâches dans l'ordre de création
par le Makefile, donc dans l'ordre où il rencontre les contraintes
de construction) :
\vspace{.5cm}


\verbatimtabinput{projectname.tasks}

\vspace{.5cm}

\item Le fichier projectname.acq (uniquement les tâches acquisition) :
\vspace{.5cm}


\verbatimtabinput{projectname.acq}

\vspace{.5cm}

\item Le fichier projectname.process (uniquement les tâches traitement) :
\vspace{.5cm}


\verbatimtabinput{projectname.process}
\ei

\section{Traduction des contraintes pour les logiciels acquisition et traitement}
On prend la liste des tâches et on traduit ce que doit faire chaque logiciel. 
\subsection{Acquisition}
On prend les tâches une à une et on voit ce qu'on doit en faire :

\bi

\item NEWACQ Seq IMAGETYP=Bias

Le logiciel d'acquisition propose cette tâche à l'observateur, l'observateur règle
le temps d'exposition et le nombre d'images ; à l'exécution, chaque image est enregistrée
avec les mots-clés FITS :

IMAGETYP='Bias'
SEQNUM = numéro de l'image dans la séquence

peut-être, si c'est utile (ça peut l'être pour des raisons d'efficacité)  quelque chose comme :

FREQ = ALL

et optionnellement : 

PROJECT = 'projectname'

Dans la suite on ne rementionnera pas la possibilité d'inclure ce champ, mais si on choisit 
a priori de l'implémenter, il faut le mettre dans toutes les images.
\vspace{.5cm}

\item NEWACQ Seq IMAGETYP=Dark EXPTIME=120
\vspace{.5cm}

Champs à inclure :
\be
\item IMAGETYPE='Dark'
\item EXPTIME=120
\item FREQ='ALL'
\item SEQNUM=numéro d'image
\ee

Pour tous les darks on fait la même chose, avec un champ EXPTIME adapté.

\vspace{.5cm}
On passe aux flats, le champ FREQ devient pertinent :
\vspace{.5cm}


\item NEWACQ Seq IMAGETYP=Flat FREQ=A 
\vspace{.5cm}

Champs à inclure :
\be
\item IMAGETYPE='Flat'
\item FREQ='A'
\item SEQNUM=numéro d'image
\ee
\vspace{.5cm}

on fait la même chose pour tous les flats, en adaptant le champ FREQ.
\vspace{.5cm}

On passe aux lights :
\vspace{.5cm}
\item NEWACQ Seq IMAGETYP=Light EXPTIME=300 FREQ=L
\vspace{.5cm}

Champs à inclure  en sortie :
\vspace{.5cm}
\be
\item IMAGETYPE='Light'
\item EXPTIME=300
\item FREQ='L'
\item SEQNUM=numéro d'image
\ee
on fait la même chose pour tous les lights, en adaptant les champs FREQ et EXPTIME.


\ei
\subsection{Traitement}
On n'impose rien sur le nom du fichier, c'est au logiciel de traitement de choisir
ce qui lui paraît pertinent comme nom. Tout ce qu'on impose c'est des mots-clés FITS
embarqués dans le fichier.
\vspace{.5cm}

\bi
\item NEWPROCESS makemaster IMAGETYP=Bias

La stratégie de  construction  de  la  séquence  par  le  logiciel  de
traitement consiste à lister les images du répertoire de travail  dont
le champ IMAGETYP vaut Bias et le champ SEQNUM est non nul. Une fois que
c'est fait, c'est terminé, le traitement est le même qu'avant.

makemaster est une instruction symbolique disant qu'il faut stacker une séquence pour en faire
un master. Dans le cas du Bias, les mots-clés à inclure dans l'image master de sortie :
\be
\item IMAGETYP='MasterBias'
\item FREQ='ALL'
\item SEQNUM = 0
les champs FREQ et SEQNUM ns sont pas nécessaires ; il peut être utile de les ajouter soit pour des raisons de cohérence soit
pour des raisons d'efficacité.
\ee
\item NEWPROCESS makemaster IMAGETYP=Dark EXPTIME=120
\vspace{.5cm}

Dans le cas du Dark, les mots-clés à inclure dans l'image master de sortie :
\vspace{.5cm}
\be
\item IMAGETYP='MasterDark'
\item EXPTIME=120
\item FREQ='ALL'
\item SEQNUM = 0
\ee
\vspace{.5cm}
Tous les darks sont traités de la même manière, avec un champ EXPTIME ajusté.
\vspace{.5cm}

\item NEWPROCESS makemaster IMAGETYP=Flat FREQ=L
En ce qui concerne les flats, le MasterBias est la seule image avec IMAGETYP=MasterBias ;
on a toutes les informations pour faire le MasterFlat L. Idem pour toutes les
fréquences.

En sortie :

\be
\item IMAGETYP='MasterFlat'
\item FREQ='L'
\item SEQNUM = 0
\ee
\vspace{.5cm}

\item NEWPROCESS makemaster IMAGETYP=Light FREQ=L EXPTIME=300

\vspace{.5cm}
En ce qui concerne les Lights, le MasterBias est la seule image avec IMAGETYP=MasterBias ;
il y a une seule image MasterDark/ExpTime=300, il y a une seule image MasterFlat/FREQ=L,
donc on a toutes les informations pour contruire le MasterLight/L/300. Idem pour toutes les
fréquences et tous les temps de pose.

En sortie :

\be
\item IMAGETYP='MasterLight'
\item FREQ='A'
\item EXPTIME='300'
\item SEQNUM = 0
\ee
\vspace{.5cm}
on fait la même chose pour tous les "Lights".

\vspace{.5cm}
Enfin pour la dernière étape, 
\vspace{.5cm}

\item NEWPROCESS composite Master\_Light\_L300 Master\_Light\_L120 Master\_Light\_L30 ...
\vspace{.5cm}

on peut simplement rendre la main à l'observateur pour la partie la plus
intéressante du traitement.


\ei
\section{Contraintes/suggestions d'implémentation}
\subsection{Spécifications}
Le format des mots-clés FITS utilisés, notamment les valeurs légitimes pour chacun d'eux 
doivent être standardisés pour autoriser l'interopérabilité des logiciels souhaitant implémenter
la proposition.

\subsection{Acquisition}
La liste des tâches d'acquisition peut être implémentée comme liste déroulante à proposer
à l'observateur. Des informations diverses sur l'état de chaque tâche peuvent accompagner cette liste, 
comme le nombre d'images déjà existantes dans telle ou telle séquence ; la sélection d'une tâche 
concernant une séquence déjà existante peut donner le choix entre ajouter des images à la séquence (SEQNUM adapté) 
ou écraser l'existant (ça nécessite de faire le ménage pour éviter des ambiguïtés au traitement).

Une tâche peut être marquée comme terminée par l'observateur, ça peut être une facilité pour le logiciel 
de traitement.

...
\subsection{Traitement}
La construction des séquences et de manière plus générale l'identification des images déjà acquises dans le 
cadre du projet est la principale nouveauté ; ça peut paraître lourd ià implémenter parce que ça implique de 
passer en revue les headers de tous les fichiers fits du répertoire de travail, mais d'une part les bénéfices 
du point de vue utilisateur sont inappréciables et d'autre part, bien organisé c'est sans doute complètement 
indolore en terme de temps de calcul par rapport au traitement lui-même.

Sur le traitement lui-même, chaque tâche du fichier .tasks doit donner lieu à une partie de script ; ce n'est pas 
forcément trivial à faire, mais toutes les informations pour produire le script final sont disponibles dans
les images et les tâches de traitement.
\end{document}
